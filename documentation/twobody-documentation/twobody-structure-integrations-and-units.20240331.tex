\documentclass[12pt%,draft
]{article}%%%%%% Choose Draft version or final
%%%%%%%%%%%%%%%%%%%%%%%%%%%%%%%%%%%%%%%%%%%%%%%%%%%%%%%%%%%%%%%%%%%%%%%%%%%%%%%
% ``He is doing his best, I am afraid.''
%                              British report card
%%%%%%%%%%%%%%%%%%%%%%%%%%%%%%%%%%%%%%%%%%%%%%%%%%%%%%%%%%%%%%%%%%%%%%%%%%%%%%%
% CONVENTIONS
%
% Grie\3hammer, Mei\3ner even in bibliography
%
% oft-used standards: EPJA
% -----------------------
% All refers are NON-capitalised:
% ref.~
% fig.~
% sect.~
% sects.~
% app.~
% eq.~
% eqs.~
% all the above NOT abbreviated but capitalised when beginning of sentence.
%
% cf.~ not italicised
% see, \textit{e.g.}, XXX
% 
% colour, polarisability, renormalisation,...
%
% \mathrm{} labels (also in reactions): particle names, frame, 
%
% punctuations inside equations are separated as "\;\;,\;\;" etc.
%%%%%%%%%%%%%%%%%%%%%%%%%%%%%%%%%%%%%%%%%%%%%%%%%%%%%%%%%%%%%%%%%%%%%%%%%%%%%%
\usepackage[british]{babel}
\selectlanguage{british}
%%%%%%%%%%%%%%%%%%%%%%%%%%%%%%%%%%%%%%%%%%%%%%%%%%%%%%%%%%%%%%%%%%%%%%%%%%%%%%%
\def\AnswerYes{y}
\def\pdflatex{y}                  %%%%%% Use pdflatex? (if not, latex)
\def\ShowLabelsVersion{n}         %%%%%% Show defs. of refs, cites or final
\def\ShowChangesVersion{n}        %%%%%% Changes highlighted or final version
\def\ShowAnnotationsVersion{n}    %%%%%% Version with annotations or final
\def\ShowFigures{y}               %%%%%% Show figures; if no: only show filename
\def\feynVersion{y}               %%%%%% Choose whether feynman graphs to be
                                  %%%%%%    generated, or .eps file included
\def\MakeArXivLinksActive{y}      %%%%%% Choose whether ArXiv links are active
%%%%%%%%%%%%%%%%%%%%%%%%%%%%%%%%%%%%%%%%%%%%%%%%%%%%%%%%%%%%%%%%%%%%%%%%%%%%%%%
%%%%%%%%%%%%%%%%%%%%%%%%%%%%%%%%%%%%%%%%%%%%%%%%%%%%%%%%%%%%%%%%%%%%%%%%%%%%%%
%%%%% for pdflatex: treating figures
\ifx\pdflatex\AnswerYes
   \usepackage[update,prepend]{epstopdf} % automaticlaly converts .eps to .pdf file
   \usepackage{grffile} % allows graphics files to have "." in filename
   \epstopdfsetup{update} % only regenerate pdf files when eps file is newer
   \DeclareGraphicsRule{*}{mps}{*}{}
\fi
%%%%% Colours in the graphs
\ifx\pdflatex\AnswerYes
    \usepackage{color}
\else
    \usepackage[dvips]{color}
\fi
%%%%% Thumbnails for PDF versions
\ifx\pdflatex\AnswerYes
    \usepackage{thumbpdf}    %
\else
    \usepackage[dvips]{thumbpdf}
\fi
%%%%%%%%%%%%%%%%%%%%%%%%%%%%%%%%%%%%%%%%%%%%%%%%%%%%%%%%%%%%%%%%%%%%%%%%%%%%
%%%% Following produces hyper-linked .pdf file
%%%% should be the very package to load, except natbib after that:
\ifx\pdflatex\AnswerYes
    \usepackage[ 
        final,
        breaklinks=true,
        colorlinks=true,           % coloured links 
        pdfborder={0 0 1},    % coloured box about link
        citecolor={blue},linkcolor={blue},urlcolor={red},
        citebordercolor={1 0 0},linkbordercolor={0 0 1},urlbordercolor={1 0 0},
        pdfpagemode=UseOutlines,% whether Acrobat opens with
                               % None or
                               % UseThumbs (thumbnails shown) or 
                               % UseOutlines (bookmarks shown) or
                               % FullScreen.
%        pagebackref,          % insert pageref to \cite{} occurence in bibitem 
        bookmarks=true,bookmarksopenlevel=4% write .pdf table of contents.
        ]{hyperref}         % Hyperlinks for PDF versions
%% Following must be used with hyperref; without it, can use ``cite'' to collapse
\else
    \usepackage[dvips,              % dvips,ps2pdf,pdftex driver 
        final,
        breaklinks=true,
        colorlinks=true,      % coloured links 
        pdfborder={0 0 1},    % coloured box about link
        citecolor={blue},linkcolor={blue},urlcolor={red},
        citebordercolor={1 0 0},linkbordercolor={0 0 1},urlbordercolor={1 0 0},
        pdfpagemode=UseOutlines,% whether Acrobat opens with
                               % None or
                               % UseThumbs (thumbnails shown) or 
                               % UseOutlines (bookmarks shown) or
                               % FullScreen.
%        pagebackref,          % insert pageref to \cite{} occurence in bibitem 
        bookmarks=true,bookmarksopenlevel=4% write .pdf table of contents.
         ]{hyperref}         % Hyperlinks for PDF versions
\fi
%%%%%%%%%%%%%%%%%%%%%%%%%%%%%%%%%%%%%%%%%%%%%%%%%%%%%%
%%%%  active links to arXiv papers, with minimal modification of SPIRES LaTeX citation:
%%%%  use \arxiv[arXiv:1234.5678 [yyyy-xx]]  or  \arxivold[yyyy-xx/1234567]  
%%%%  (yyyy-xx is hep-ph or nucl-ex...)
\ifx\MakeArXivLinksActive\AnswerYes
   \usepackage{xparse}
   \NewDocumentCommand{\arxiv} %
   {r [: u{ [} u{]]} }{[\href{http://arxiv.org/abs/#2}{arXiv:#2}~[#3]]}
   \NewDocumentCommand{\arxivold} {r[]}{[\href{http://arxiv.org/abs/#1}{#1}]}
   \NewDocumentCommand{\arXiv} %
   {r [: u{ [} u{]]} }{[\href{http://arxiv.org/abs/#2}{arXiv:#2}~[#3]]}
   \NewDocumentCommand{\arXivold} {r[]}{[\href{http://arxiv.org/abs/#1}{#1}]}
\else
%%%% commands have no effect
   \newcommand{\arxiv}[1][]{[#1]}
   \newcommand{\arxivold}[1][]{[#1]}
   \newcommand{\arXiv}[1][]{[#1]}
   \newcommand{\arXivold}[1][]{[#1]}
\fi
%%%%%%%%%%%%%%%%%%%%%%%%%%%%%%%%%%%%%%%%%%%%%%%%%%%%%%%%%%%%%%%%%%%%%%%%%%%%
\usepackage{doi} % \doi{} converts DOI numbers into clickable links
%%%%%%%%%%%%%%%%%%%%%%%%%%%%%%%%%%%%%%%%%%%%%%%%%%%%%%%%%%%%%%%%%%%%%%%%%%%%
\usepackage[numbers,sort&compress]{natbib}%%% collapses [1,2,3] -> [1-3] etc
%%%%%%%%%%%%%%%%%%%%%%%%%%%%%%%%%%%%%%%%%%%%%%%%%%%%%%%%%%%%%%%%%%%%%%%%%%%%
\usepackage{graphicx}            %%% for pictures and figures
\graphicspath{{figures/}{./}}

\usepackage{multirow}                   %%% tabs over multiple rows
\usepackage{amssymb}
\usepackage{amsmath}
\usepackage{bbm} % double-line symbols (RR), with numerals (11) and lower-case
\usepackage{xspace}                    %%% correct spacing self-defined macros
\usepackage{dcolumn}                    %%% Align table columns on decimal point
\usepackage{pbox}

\usepackage{bm}                        %%% bold math

\usepackage{slashed}
% the \slashed{}-command produces a Feynman dagger through a letter.
% \declareslashed{#1}{#2}{#3}{#4}{#5}:
%#1 normally empty, can be \mathop etc; #2 normally /
% #3 right shift; #4 up shift; #5 symbol
%%%%%%%%%%%%%%%%%%%%%%%%%%%%%%%%%%%
%%%%%%%%%%%%%%%%%%%%%%%%%%%%%%%%%%%%%%%%%%%%%%%%%%%%%%%%%%%%%%%%%%%%%%%%%%%%
\ifx\feynVersion\AnswerYes
   \usepackage{feynmp-auto} % for Feynman diagrams; nice definitions after \begin{document}
\fi
%%%%%%%%%%%%%%%%%%%%%%%%%%%%%%%%%%%%%%%%%%%%%%%%%%%%%%%%%%%%%%%%%%%%%%%%%%%%

\setlength{\unitlength}{1pt}              %%% always good to have...

\usepackage{chngcntr} % allows one to reset counters between sections etc
\counterwithin*{equation}{section} % reset equation counter with each section
\renewcommand{\theequation}{\thesection.\arabic{equation}}
\renewcommand{\labelenumi}{(\arabic{enumi})}
\renewcommand{\labelitemi}{--}
\renewcommand{\arraystretch}{1.5}

% less space between \paragraph headings
\makeatletter
\renewcommand{\paragraph}{%
  \@startsection{paragraph}{4}%
  {\z@}{1.5ex \@plus 1ex \@minus .2ex}{-1em}%
  {\normalfont\normalsize\bfseries}%
}
\makeatother

%%%%%%%%%%%%%%%%%%%%%%%%%%%%%%%%%%%%%%%%%%%%%%%%%%%%%%%%%%%%%%%%%%%%%%%%%%%%%%%
% text: dimensions for A4 and letter alike

\textheight22cm
\textwidth16.4cm
\topmargin = -1.5 true cm
\addtolength{\evensidemargin}{-1.25cm}
\addtolength{\oddsidemargin}{-1.25cm}
\renewcommand{\textfraction}{0}

\flushbottom                               %%% footnotes inside page

% Necessary new hyphenations
\hyphenation{Z-para-meter-isa-tion re-scat-ter-ing Re-scat-ter-ing}

%%%%%%%%%%%%%%%%%%%%%%%%%%%%%%%%%%%%%%%%%%%%%%%%%%%%%%%%%%%%%%%%%%%%%%%%%%%%%%%
%%%% Draft Versions
%%%%
%%%% For Draft version showing labels
\ifx\ShowLabelsVersion\AnswerYes
% shows definitions of labels, refs, cites etc.
   \usepackage[color]{showkeys} 
   \definecolor{refkey}{gray}{.5}   % slightly gray font
   \definecolor{labelkey}{gray}{.5} % slightly gray font
   \renewcommand{\showkeyslabelformat}[1]{\fbox{\normalfont\tiny\ttfamily#1}}
   % very small font for labels
\fi
%%%%%%%%%%%%%%%%%%%%%%%%%%%%%%%%%%%%%%%%%%%%%%%%%%%%%%%%%%%%%%%%%%%%%%%%%%%%%%%
%%%% For Draft version showing annotations
\ifx\ShowAnnotationsVersion\AnswerYes
% highlights changes and shows replaced text
   \typeout{*** Draft version: annotations shown ***}
% Comments
   \newcommand{\comment}[1]{{\scriptsize\sffamily\bfseries{#1}}}
   \newcommand{\margin}[1]{\marginpar{\scriptsize\sffamily\bfseries{#1}}}
   \pagestyle{myheadings}
   \markboth{\textbf{Draft version \today}}{\textbf{Draft version \today}}
   \newcommand{\drafty}{\textbf{Draft version \today} \hfill}
\else
   \newcommand{\comment}[1]{}
   \newcommand{\margin}[1]{}
   \pagestyle{myheadings}
   \markboth{\textbf{Version \today}}{\textbf{Version \today}}
   \newcommand{\drafty}{\hspace*{\fill}\textbf{Version \today}\hspace*{\fill}}
\fi
%%%%%%%%%%%%%%%%%%%%%%%%%%%%%%%%%%%%%%%%%%%%%%%%%%%%%%%%%%%%%%%%%%%%%%%%%%%%%%%
%%%% For Draft version showing changes
\ifx\ShowChangesVersion\AnswerYes
% highlights changes and shows replaced text
   \typeout{*** Draft version: changes shown ***}
   \usepackage{ulem}
   % provides underlines of various styles, good for commenting etc.
   % \uline{important}    underlined text
   % \dashuline{dashing}  dashed underline
   % \dotuline{dotty}     dotted underline
   % \uuline{urgent}      double-underlined text
   % \uwave{boat}         wavy underline 
   % \sout{wrong}         line drawn _through_ word
   % \xout{removed}       marked over (many "/////")
   \newcommand{\replace}[2]{\sout{\protect#1}\textbf{#2}} 
   % replace #1 (strike-out) with #2 (highlighted)
   \newcommand{\add}[1]{\textbf{#1}}               % insert #1 (highlighted)
   \newcommand{\delete}[1]{\sout{#1}}            % delete #1 (strike-out)
   \renewcommand{\emph}[1]{\textit{#1}}           % ulem overwrites def of \emph as
                                            % textit, so reinstate here
\else
   \newcommand{\uline}[1]{#1}
   \newcommand{\dashuline}[1]{#1}
   \newcommand{\dotuline}[1]{#1}
   \newcommand{\uuline}[1]{#1}
   \newcommand{\uwave}[1]{#1}
   \newcommand{\sout}[1]{}
   \newcommand{\xout}[1]{}
   \newcommand{\replace}[2]{#2}
   \newcommand{\add}[1]{#1}
   \newcommand{\delete}[1]{}
\fi

%%%%%%%%%%%%%%%%%%%%%%%%%%%%%%%%%%%%%%%%%%%%%%%%%%%%%%%%%%%%%%%%%%%%%%%%%%%%%%%
%%%%%%%%%%%%%%%%%%%%%%%%%%%%%%%%%%%%%%%%%%%%%%%%%%%%%%%%%%%%%%%%%%%%%%%%%%%%%%%
% "Stylistic" Definitions
%
\newcommand{\3}{\ss}
\newcommand{\disc}{\discretionary{}{}{}}%makes optional hyphenation without "-"
\newcommand{\absatz}{\vspace{2ex}\noindent}

% eliminate colours
\newcommand{\blue}[1]{#1}
\newcommand{\red}[1]{#1}
\newcommand{\black}[1]{#1}
\newcommand{\green}[1]{#1}

% define abbreviations

\newcommand{\eg}{\textit{e.g.}\xspace}
\newcommand{\ie}{\textit{i.e.}\xspace}
\newcommand{\etal}{\textit{et al.}\xspace}

%%%%%%%%%%%%%%%%%%%%%%%%%%%%%%%%%%%%%%%%%%%%%%%%%%%%%%%%%%%%%%%%%%%%%%%%%%%%%%%
% Mathematical Defs.
%

\newcommand{\dis}{\displaystyle}
\newcommand{\txt}{\textstyle}
\newcommand{\script}{\scriptstyle}
\newcommand{\fs}{\scriptstyle} % adjusts size of labels e.g. in feynmf-diagrams
\newcommand{\non}{\nonumber}
\newcommand{\hf}{\hspace*{\fill}}
\newcommand{\hq}{\hspace{0.5em}}
\newcommand{\hqq}{\hspace{1em}}
\newcommand{\hqqq}{\hspace{2em}}
\newcommand{\hqm}{\hspace*{-0.25em}}
\newcommand{\hqmm}{\hspace*{-0.5em}}
\newcommand{\hqmmm}{\hspace*{-1.0em}}

\newcommand{\half}{\frac{1}{2}}
\newcommand{\e}{\mathrm{e}}
\newcommand{\ii}{\mathrm{i}}
\newcommand{\dd}{\mathrm{d}}
\newcommand{\tr}{\mathrm{tr}}
\newcommand{\T}{\mathrm{T}}

\newcommand{\deint}[2]{\dd^{#1}\! #2\;}
\newcommand{\deintdim}[2]{\frac{\dd^{#1}\! #2}{(2\pi)^{#1}}\;}
\newcommand{\dedreix}{\dd^{3}\:\!\! x\;}
\newcommand{\dedreiy}{\dd^{3}\:\!\! y\;}

\newcommand{\de}{\partial}

\renewcommand{\deg}{\circ}

% The following construct adds a bit of space to the end of a vector, so that
% the vector arrows do not collide with subscript like ² and '  
\newcommand{\vectorwithspace}[1]{\vec{#1}\mkern2mu\vphantom{#1}}
\newcommand{\vect}[1]{\vectorwithspace{#1}}

\newcommand{\grad}{\vectorwithspace{\nabla}}
\newcommand{\dev}{\vectorwithspace{\de}}

\newcommand{\av}{\vectorwithspace{a}}
\newcommand{\bv}{\vectorwithspace{b}}
\newcommand{\cv}{\vectorwithspace{c}}
\newcommand{\dv}{\vectorwithspace{d}}
\newcommand{\ev}{\vectorwithspace{e}}
\newcommand{\fv}{\vectorwithspace{f}}
\newcommand{\gv}{\vectorwithspace{g}}
\newcommand{\hv}{\vectorwithspace{h}}
\newcommand{\iv}{\vectorwithspace{i}}
\newcommand{\jv}{\vectorwithspace{j}}
\newcommand{\kv}{\vectorwithspace{k}}
\newcommand{\lv}{\vectorwithspace{l}}
\newcommand{\mv}{\vectorwithspace{m}}
\newcommand{\nv}{\vectorwithspace{n}}
\newcommand{\ov}{\vectorwithspace{o}}
\newcommand{\pv}{\vectorwithspace{p}}
\newcommand{\qv}{\vectorwithspace{q}}
\newcommand{\rv}{\vectorwithspace{r}}
\newcommand{\sv}{\vectorwithspace{s}}
\newcommand{\tv}{\vectorwithspace{t}}
\newcommand{\uv}{\vectorwithspace{u}}
\newcommand{\vv}{\vectorwithspace{v}}
\newcommand{\wv}{\vectorwithspace{w}}
\newcommand{\xv}{\vectorwithspace{x}}
\newcommand{\yv}{\vectorwithspace{y}}
\newcommand{\zv}{\vectorwithspace{z}}
\newcommand{\Av}{\vectorwithspace{A}}
\newcommand{\Bv}{\vectorwithspace{B}}
\newcommand{\Cv}{\vectorwithspace{C}}
\newcommand{\Dv}{\vectorwithspace{D}}
\newcommand{\Ev}{\vectorwithspace{E}}
\newcommand{\Fv}{\vectorwithspace{F}}
\newcommand{\Gv}{\vectorwithspace{G}}
\newcommand{\Hv}{\vectorwithspace{H}}
\newcommand{\Iv}{\vectorwithspace{I}}
\newcommand{\Jv}{\vectorwithspace{J}}
\newcommand{\Kv}{\vectorwithspace{K}}
\newcommand{\Lv}{\vectorwithspace{L}}
\newcommand{\Mv}{\vectorwithspace{M}}
\newcommand{\Nv}{\vectorwithspace{N}}
\newcommand{\Ov}{\vectorwithspace{O}}
\newcommand{\Pv}{\vectorwithspace{P}}
\newcommand{\Qv}{\vectorwithspace{Q}}
\newcommand{\Rv}{\vectorwithspace{R}}
\newcommand{\Sv}{\vectorwithspace{S}}
\newcommand{\Tv}{\vectorwithspace{T}}
\newcommand{\Uv}{\vectorwithspace{U}}
\newcommand{\Vv}{\vectorwithspace{V}}
\newcommand{\Wv}{\vectorwithspace{W}}
\newcommand{\Xv}{\vectorwithspace{X}}
\newcommand{\Yv}{\vectorwithspace{Y}}
\newcommand{\Zv}{\vectorwithspace{Z}}

\newcommand{\Complexes}{\mathbb{C}}
\newcommand{\Imaginaries}{\mathbb{I}}
\newcommand{\Integers}{\mathbb{Z}}
\newcommand{\Naturals}{\mathbb{N}}
\newcommand{\Rationals}{\mathbb{Q}}
\newcommand{\Reals}{\mathbb{R}}


\renewcommand{\Re}{\mathrm{Re}}
\renewcommand{\Im}{\mathrm{Im}}
\newcommand{\bra}{\langle}
\newcommand{\ket}{\rangle}

\newcommand{\Folgt}{\Longrightarrow}

% Eff. Nuclear Theory

\newcommand{\mpi}{\ensuremath{m_\pi}}
\newcommand{\mpiphys}{\ensuremath{m_{\pi\text{phys}}}}
\newcommand{\fpi}{\ensuremath{f_\pi}}
\newcommand{\MeV}{\ensuremath{\mathrm{MeV}}}
\newcommand{\fm}{\ensuremath{\mathrm{fm}}}
\newcommand{\ChiEFT}{$\chi$EFT\xspace}
\newcommand{\EFTNoPion}{EFT($\slashed{\pi}$)\xspace}
\newcommand{\NoPion}{\ensuremath{\slashed{\pi}}}
\newcommand{\upNoPion}{\ensuremath{{}^\slashed{\pi}}}
\newcommand{\downNoPion}{\ensuremath{{}_\slashed{\pi}}}
\newcommand{\LambdaChi}{\ensuremath{\overline{\Lambda}_\chi}}
\newcommand{\LambdaNoPion}{\ensuremath{\Lambda_\slashed{\pi}}}
\newcommand{\QNoPion}{\ensuremath{Q_\slashed{\pi}}}

\newcommand{\NXLO}[1]{N\ensuremath{{}^{#1}}LO\xspace}
\newcommand{\NtwoLO}{\NXLO{2}}
\newcommand{\NthreeLO}{\NXLO{3}}

\newcommand{\wave}[3]{\ensuremath{{}^{#1}\mathrm{#2}_{#3}}}
\newcommand{\oneS}{\wave{1}{S}{0}}
\newcommand{\twoS}{\wave{2}{S}{\half}}
\newcommand{\threeS}{\wave{3}{S}{1}}
\newcommand{\fourS}{\wave{4}{S}{\frac{3}{2}}}

\newcommand{\HIGS}{HI$\gamma$S\xspace}
\newcommand{\threeH}{\ensuremath{{}^3}H\xspace}
\newcommand{\threeHe}{\ensuremath{{}^3}He\xspace}
\newcommand{\fourHe}{\ensuremath{{}^4}He\xspace}

% Definitions for Polarisabilities 

\newcommand{\alphae}{\ensuremath{\alpha_{E1}}}
\newcommand{\betam}{\ensuremath{\beta_{M1}}}
\newcommand{\gammaee}{\ensuremath{\gamma_{E1E1}}}
\newcommand{\gammamm}{\ensuremath{\gamma_{M1M1}}}
\newcommand{\gammaem}{\ensuremath{\gamma_{E1M2}}}
\newcommand{\gammame}{\ensuremath{\gamma_{M1E2}}}
\newcommand{\gammazero}{\ensuremath{\gamma_{0}}}
\newcommand{\gammapi}{\ensuremath{\gamma_{\pi}}}
\newcommand{\alphaep}{\ensuremath{\alpha_{E1}^{(\mathrm{p})}}}
\newcommand{\betamp}{\ensuremath{\beta_{M1}^{(\mathrm{p})}}}
\newcommand{\gammaeep}{\ensuremath{\gamma_{E1E1}^{(\mathrm{p})}}}
\newcommand{\gammammp}{\ensuremath{\gamma_{M1M1}^{(\mathrm{p})}}}
\newcommand{\gammaemp}{\ensuremath{\gamma_{E1M2}^{(\mathrm{p})}}}
\newcommand{\gammamep}{\ensuremath{\gamma_{M1E2}^{(\mathrm{p})}}}
\newcommand{\gammazerop}{\ensuremath{\gamma_{0}^{(\mathrm{p})}}}
\newcommand{\gammapip}{\ensuremath{\gamma_{\pi}^{(\mathrm{p})}}}
\newcommand{\alphaen}{\ensuremath{\alpha_{E1}^{(\mathrm{n})}}}
\newcommand{\betamn}{\ensuremath{\beta_{M1}^{(\mathrm{n})}}}
\newcommand{\gammaeen}{\ensuremath{\gamma_{E1E1}^{(\mathrm{n})}}}
\newcommand{\gammammn}{\ensuremath{\gamma_{M1M1}^{(\mathrm{n})}}}
\newcommand{\gammaemn}{\ensuremath{\gamma_{E1M2}^{(\mathrm{n})}}}
\newcommand{\gammamen}{\ensuremath{\gamma_{M1E2}^{(\mathrm{n})}}}
\newcommand{\gammazeron}{\ensuremath{\gamma_{0}^{(\mathrm{n})}}}
\newcommand{\gammapin}{\ensuremath{\gamma_{\pi}^{(\mathrm{n})}}}
\newcommand{\alphaes}{\ensuremath{\alpha_{E1}^{(\mathrm{s})}}}
\newcommand{\betams}{\ensuremath{\beta_{M1}^{(\mathrm{s})}}}
\newcommand{\gammaees}{\ensuremath{\gamma_{E1E1}^{(\mathrm{s})}}}
\newcommand{\gammamms}{\ensuremath{\gamma_{M1M1}^{(\mathrm{s})}}}
\newcommand{\gammaems}{\ensuremath{\gamma_{E1M2}^{(\mathrm{s})}}}
\newcommand{\gammames}{\ensuremath{\gamma_{M1E2}^{(\mathrm{s})}}}
\newcommand{\gammazeros}{\ensuremath{\gamma_0^{(\mathrm{s})}}}
\newcommand{\gammapis}{\ensuremath{\gamma_\pi^{(\mathrm{s})}}}
\newcommand{\alphaev}{\ensuremath{\alpha_{E1}^{(\mathrm{v})}}}
\newcommand{\betamv}{\ensuremath{\beta_{M1}^{(\mathrm{v})}}}
\newcommand{\gammaeev}{\ensuremath{\gamma_{E1E1}^{(\mathrm{v})}}}
\newcommand{\gammammv}{\ensuremath{\gamma_{M1M1}^{(\mathrm{v})}}}
\newcommand{\gammaemv}{\ensuremath{\gamma_{E1M2}^{(\mathrm{v})}}}
\newcommand{\gammamev}{\ensuremath{\gamma_{M1E2}^{(\mathrm{v})}}}
\newcommand{\gammazerov}{\ensuremath{\gamma_{0}^{(\mathrm{v})}}}
\newcommand{\gammapiv}{\ensuremath{\gamma_{\pi}^{(\mathrm{v})}}}

\newcommand{\muv}{\ensuremath{\mu^{(\mathrm{v})}}}

\newcommand{\kappas}{\ensuremath{\kappa^{(\mathrm{s})}}}
\newcommand{\kappav}{\ensuremath{\kappa^{(\mathrm{v})}}}
\newcommand{\kappap}{\ensuremath{\kappa^{(\mathrm{p})}}}

\newcommand{\piN}{\pi\mathrm{N}}
\newcommand{\gammaN}{\gamma \mathrm{N}}
\newcommand{\gammaonp}{\gamma\mathrm{p}}
\newcommand{\p}{\mathrm{p}}
\newcommand{\n}{\mathrm{n}}
\newcommand{\N}{\mathrm{N}}

\newcommand{\MN}{\ensuremath{M_\mathrm{N}}} % nucleon mass
\newcommand{\Mp}{\ensuremath{M_\mathrm{p}}} % proton mass
\newcommand{\Mn}{\ensuremath{M_\mathrm{n}}} % neutron mass
\newcommand{\Md}{\ensuremath{M_\mathrm{d}}} % deuteron mass
\newcommand{\MHe}{\ensuremath{M_\text{\threeHe}}} % deuteron mass
\newcommand{\MDelta}{\ensuremath{M_\Delta}} % Delta mass
\newcommand{\DeltaM}{\ensuremath{\Delta_{\scriptscriptstyle M}}} % Delta-nucleon mass splitting
\newcommand{\wpi}{\ensuremath{\omega_\pi}}
\newcommand{\omegacm}{\ensuremath{\omega_\mathrm{cm}}}
\newcommand{\thetacm}{\ensuremath{\theta_\mathrm{cm}}}

\newcommand{\ga}{g_{\scriptscriptstyle A}}
\newcommand{\gpiNN}{g_{\pi{\scriptscriptstyle\text{NN}}}}
\newcommand{\gpiNDelta}{g_{\pi{\scriptscriptstyle\text{N}\Delta}}}
\newcommand{\lambdachi}{\Lambda_\chi}
\newcommand{\ChPT}{\ensuremath{\chi \mathrm{PT}}}
\newcommand{\alphaEM}{\alpha_{\scriptscriptstyle\text{EM}}}

\newcommand{\gE}{\ensuremath{g_{\scriptscriptstyle E}}}
\newcommand{\gM}{\ensuremath{g_{\scriptscriptstyle M}}}


% Defining the expansion parameter

% Definition of all the nice cal-letters

\newcommand{\calA}{\mathcal{A}} \newcommand{\calB}{\mathcal{B}}
\newcommand{\calC}{\mathcal{C}} \newcommand{\calD}{\mathcal{D}}
\newcommand{\calE}{\mathcal{E}} \newcommand{\calF}{\mathcal{F}}
\newcommand{\calG}{\mathcal{G}} \newcommand{\calH}{\mathcal{H}}
\newcommand{\calI}{\mathcal{I}} \newcommand{\calJ}{\mathcal{J}}
\newcommand{\calK}{\mathcal{K}} \newcommand{\calL}{\mathcal{L}}
\newcommand{\calM}{\mathcal{M}} \newcommand{\calN}{\mathcal{N}}
\newcommand{\calO}{\mathcal{O}} \newcommand{\calP}{\mathcal{P}}
\newcommand{\calQ}{\mathcal{Q}} \newcommand{\calR}{\mathcal{R}}
\newcommand{\calS}{\mathcal{S}} \newcommand{\calT}{\mathcal{T}}
\newcommand{\calU}{\mathcal{U}} \newcommand{\calV}{\mathcal{V}}
\newcommand{\calW}{\mathcal{W}} \newcommand{\calX}{\mathcal{X}}
\newcommand{\calY}{\mathcal{Y}} \newcommand{\calZ}{\mathcal{Z}}

% Defining one's own title page

\newcommand{\mytitle}[1]{\begin{center}\LARGE{\textbf{#1}}\end{center}}
\newcommand{\myauthor}[1]{\textbf{#1}}
\newcommand{\myaddress}[1]{\textit{#1}}
\newcommand{\mypreprint}[1]{\begin{flushright}#1\end{flushright}}

% Defining note command for reference to manu-script
\newcommand{\note}[1]{\emph{[#1]}}

%%%%%%%%%%%%%%%%%%%%%%%%%%%%%%%%%%%%%%%%%%%%%%%%%%%%%%%%%%%%%%%%%%%%%%%%%%%%%%%
%%%%% fixes for this paper

\newcommand{\wf}{}%{\Psi,}    % ket
\newcommand{\wfbra}{}%{\Psi,} % bra
\newcommand{\sepp}{,}
\newcommand{\sep}{}

\newcommand{\jrel}{\ensuremath{j_{12}}}

\usepackage{xstring}% so that can insert "()" between arguments if we have sums
\newcommand{\CG}[6]{\langle {#1} {#2}%\;
  {\IfSubStr{#4}{+}{(#4)}{\IfSubStr{#4}{-}{(#4)}{#4}}} 
  {\IfSubStr{#5}{+}{(#5)}{\IfSubStr{#5}{-}{(#5)}{#5}}}  | 
  {#3} {\IfSubStr{#6}{+}{(#6)}{\IfSubStr{#6}{-}{(#6)}{#6}}} \rangle}

%%%%%%%%%%%%%%%%%%%%%%%%%%%%%%%%%%%%%%%%%%%%%%%%%%%%%%%%%%%%%%%%%%%%%%%%%%%%%%%
%
% Further Comments:
%
% TAKE SPIRES LaTeX(US) STYLE
%
%%%%%%%%%%%%%%%%%%%%%%%%%%%%%%%%%%%%%%%%%%%%%%%%%%%%%%%%%%%%%%%%%%%%%%%%%%%%%%%
%%%%%%%%%%%%%%%%%%%%%%%%%%%%%%%%%%%%%%%%%%%%%%%%%%%%%%%%%%%%%%%%%%%%%%%%%%%%%%%
%%%%%%%%%%%%%%%%%%%%%%%%%%%%%%%%%%%%%%%%%%%%%%%%%%%%%%%%%%%%%%%%%%%%%%%%%%%%%%%

\begin{document}
%%%%%%%%%%%%%%%%%%%%%%%%%%%%%%%%%%%%%%%%%%%%%%%%%%%%%%%%%%%%%%%%%%%%%%%%%%%%%%%
% The following is used only if the Feynman diagrams are to be generated
% explicitly:
\ifx\feynVersion\AnswerYes
\begin{fmffile}{figuresfeyn}
  
  \fmfset{curly_len}{2mm} \fmfset{dash_len}{1.5mm} \fmfset{wiggly_len}{3mm}
% the following sets the galleries on which the external verices lie as 
% straight lines, not curved. 
  \fmfstraight

\newcommand{\feynbox}[2]{\mbox{\black{\parbox{#1}{#2}}}}

\fmfcmd{color green,truegreen; green:=(0 , .75 , 0); truegreen:=(0 , 1 , 0);}

% invisible frame around fmfgraph*-environment:
% #1: Optional text_after_ graph, e.g. \newline and figure label (a) etc. 
% #2: x-size of fmfgraph* in \unitlength
% #3: y-size of fmfgraph* in \unitlength
% #4: additional space on LEFT of graph in \unitlength  
% #5: additional space on TOP of graph in \unitlength  
% #6: additional space on RIGHT of graph in \unitlength  
% #7: additional space on BOTTOM of graph in \unitlength  
% #8: the graph
%
% USAGE: 
% \feyngraph[\\\protect\raisebox[-1ex}{\protect\centerline{(a)}}]
%           {xsize}{ysize}{left}{top}{right}{bottom}{graph}  
%  
% If second and next-to-last line are commented out, a box is put around the
% invisible frame: Good for proof-reading.
%
\newlength{\feyngraphlength}
\newcommand{\feyngraph}[8][]{%
  \setlength{\feyngraphlength}{#2\unitlength}%
  \addtolength{\feyngraphlength}{#4\unitlength}%
  \addtolength{\feyngraphlength}{#6\unitlength}%
  \parbox[c]{\feyngraphlength} {%
   %\framebox{%%%%%
      \fmfframe(#4,#5)(#6,#7){%
        \begin{fmfgraph*}(#2,#3)%
          #8%
        \end{fmfgraph*}%
        }#1%
      }%
    %}%%%%%
   }

%  The following produces an ellipse along a line given by two vertices.
%  Application: \fmf{ellipse,foreground=red,rubout=1}{v1,v2}
%               draws ellipse with major semi-axis 0.6*(distance v1-v2) and
%                                  minor semi-axis 0.3*(distance v1-v2). 
%               around centre of {v1,v2}
    \fmfcmd{vardef ellipseraw (expr p, ang) = save radx; numeric radx;
      radx=6/10 length p; save rady; numeric rady; rady=3/10 length p; pair
      center; center:=point 1/2 length(p) of p; save t; transform t;
      t:=identity xscaled (2*radx*h) yscaled (2*rady*h) rotated (ang + angle
      direction length(p)/2 of p) shifted center; fullcircle transformed t
      enddef;
                %
      style_def ellipse expr p= shadedraw ellipseraw (p,0); enddef; }
%  The following produces an ellipse along a line given by two vertices,
%    hatched
%  Application: \fmf{hatchedellipse,foreground=red,rubout=1}{v1,v2}
%               draws ellipse with major semi-axis 0.6*(distance v1-v2) and
%                                  minor semi-axis 0.3*(distance v1-v2). 
%               around centre of {v1,v2}
    \fmfcmd{ style_def hatchedellipse expr p= hatchdraw ellipseraw (p,0);
      enddef; }
%  The following produces an ellipse along a line given by two vertices.
%  Application: \fmf{newellipse,foreground=red,rubout=1}{v1,v2}
%               draws ellipse with major semi-axis 0.6*(distance v1-v2) and
%                                  minor semi-axis 0.3*(distance v1-v2)
%               around centre of {v1,v2}
    \fmfcmd{vardef newellipseraw (expr p, ang) = save radx; numeric radx;
      radx=8/10 length p; save rady; numeric rady; rady=4/10 length p; pair
      center; center:=point 1/2 length(p) of p; save t; transform t;
                t:=identity xscaled (radx*h) %here the difference to ellipse
                yscaled (rady*h)             %here the difference to ellipse
                rotated (ang + angle direction length(p)/2 of p) shifted
                center; fullcircle transformed t enddef;
                %
                style_def newellipse expr p= shadedraw newellipseraw (p,0);
                enddef; }
%  The following produces a circle along a line given by two vertices.
%  Application: \fmf{mycircle,foreground=red,rubout=1}{v1,v2}
%               draws circle with radius 0.1*(distance v1-v2)
%               around centre of {v1,v2}
              \fmfcmd{vardef mycircleraw (expr p, ang) = save radx; numeric
                radx; radx=1/10 length p; save rady; numeric rady; rady=1/10
                length p; pair center; center:=point 1/2 length(p) of p; save
                t; transform t; t:=identity xscaled (2*radx*h) yscaled
                (2*rady*h) rotated (ang + angle direction length(p)/2 of p)
                shifted center; fullcircle transformed t enddef;
                %
                style_def mycircle expr p= shadedraw mycircleraw (p,0);
                enddef; }
%  The following produces a semi-ellipse along a line given by two vertices.
%  Application: \fmf{(left/right)semicircle,foreground=red,rubout=1}{v1,v2}
%               draws semi-circle with major semi-axis 0.6*(distance v1-v2) and
%                                  minor semi-axis 0.3*(distance v1-v2)
%               around centre of {v1,v2}
%  leftsemicircle: decresecnt moon; rightsemicircle: crescent moon.
              \fmfcmd{vardef mysemicircleraw (expr p, ang) = save radx;
                numeric radx; radx=6/10 length p; save rady; numeric rady;
                rady=4/10 length p; pair center; center:=point 1/2 length(p)
                of p; save t; transform t; t:=identity xscaled (2*radx*h)
                yscaled (2*rady*h) rotated (ang + angle direction length(p)/2
                of p) shifted center; halfcircle transformed t enddef;
                %
                style_def leftsemicircle expr p= shadedraw mysemicircleraw
                (p,0)--cycle enddef; style_def rightsemicircle expr p=
                shadedraw mysemicircleraw (p,180)--cycle enddef; }
%  The following produces a triple line. Improvable.
              \fmfcmd{style_def triple expr p = save oldpen; pen oldpen;
                oldpen := currentpen; pickup oldpen scaled 5; ccutdraw p;
                pickup oldpen scaled 3; cullit; draw p withcolor white;
                cullit; pickup oldpen scaled 1; cullit; ccutdraw p; cullit;
                enddef;}
%
    %%%%%%%%%%%%%%%%%%%%%%%%%%%%%%%%%%%%%%%%%%%
\fmfcmd{vardef  mysemicircleraw (expr p, ang) =
                save radx;
                numeric radx;
                radx=4/10 length p;%was 6/10 originally....
                save rady;
                numeric rady;
                rady=4/10 length p;
                pair center;
                center:=point 1/2 length(p) of p;
                save t;
                transform t;
                t:=identity xscaled (2*radx*h)
                yscaled (2*rady*h)
                rotated (ang + angle direction length(p)/2 of p)
                shifted center;
                halfcircle transformed t
                enddef;
                %
                style_def leftsemicircle expr p=
                shadedraw mysemicircleraw (p,0)--cycle
                enddef;
                style_def rightsemicircle expr p=
                shadedraw mysemicircleraw (p,180)--cycle
                enddef;
                }
\fi
    %%%%%%%%%%%%%%%%%%%%%%%%%%%%%%%%%%%%%%%%%%%
% End of part which is active when Feynman graphs to be drawn explicitly.
%%%%%%%%%%%%%%%%%%%%%%%%%%%%%%%%%%%%%%%%%%%%%%%%%%%%%%%%%%%%%%%%%%%%%%%%%%%%%%%
%%%%%%%%%%%%%%%%%%%%%%%%%%%%%%%%%%%%%%%%%%%%%%%%%%%%%%%%%%%%%%%%%%%%%%%%%%%%%%%
%%%%%%%%%%%%%%%%%%%%%%%%%%%%%%%%%%%%%%%%%%%%%%%%%%%%%%%%%%%%%%%%%%%%%%%%%%%%%%%
\section*{Notes On the Twobody Density Code Structure: Units and Integrations}
\label{sec:}


\emph{Harald W.~Grie\3hammer}

\absatz
Institute for Nuclear Studies, Department of Physics,
George Washington University, Washington DC 20052, USA

\absatz Based on hgrie's and Alex Long's research of the code, these notes are
expanded from my notes in hgrie's notebook \note{Few-N Processes With Densities,
pp.~15-21}. Equation and page numbers in \emph{emphasis} refer to this
notebook. In addition, there are comments in the code and emails Nov 2023.

Text in \texttt{texttt} refers to code variable, file names, routines, etc.

%%%%%%%%%%%%%%%%%%%%%%%%%%%%%%%%%%%%%%%%%%%%%%%%%%%%%%%%%%%%%%%%%%%%%%%%%%%%%%%
%%%%%%%%%%%%%%%%%%%%%%%%%%%%%%%%%%%%%%%%%%%%%%%%%%%%%%%%%%%%%%%%%%%%%%%%%%%%%%%
%%%%%%%%%%%%%%%%%%%%%%%%%%%%%%%%%%%%%%%%%%%%%%%%%%%%%%%%%%%%%%%%%%%%%%%%%%%%%%%
% Main Body
%

\ref{eq} \comment{nonsense} This is \replace{wrong}{right} \add{this is added
  text}
\delete{no text}


\drafty

\tableofcontents

\newpage

%%%%%%%%%%%%%%%%%%%%%%%%%%%%%%%%%%%%%%%%%%%%%%%%%%%%%%%%%%%%%%%%%%%%%%%%%%%%%%%
\section{Purpose}
\label{sec:purpose}

An explanation of the twobody code structure with emphasis on integrations and
units.

%%%%%%%%%%%%%%%%%%%%%%%%%%%%%%%%%%%%%%%%%%%%%%%%%%%%%%%%%%%%%%%%%%%%%%%%%%%%%%%
\section{Units}
\label{sec:units}


The "mantle" code has base unit $\fm$ (but NOT the output
file/\texttt{Result()}, see below!):

\begin{tabular}{ll}
  \texttt{p12, p12p} & momenta in $\fm^{-1}$\\
  \texttt{rho}       & 2N density in $\fm^3$ (quantum numbers per volume
                       momentum space)\\
  \texttt{k}        &photon omentum/energy still given in $\MeV$
\end{tabular}

However, in \texttt{finalstatesums.twobodyvia2Ndensity.f}, the call

\hqqq         \texttt{call
  Calculate2BIntegralI2(...,p12*HC,P12MAG(ip12p)*HC,...)}

converts the momenta from $\fm^{-1}$ to $\MeV$. That subroutine is defined in \texttt{calculate2BI2.f}.
Therefore, that routine and the subsequent "kernel" parts of the code use base
unit $\MeV$:

\begin{tabular}{ll}
  \texttt{calculate2BI2.f}&\\
  \texttt{2Bkernel.*.f} & part of "kernel", specific to process\\
  \texttt{2Bspinsym.*.f}         & part of "kernel", specific to process\\
  \texttt{2Bspinasym.*.f}      & part of "kernel", specific to process
\end{tabular}
Therefore, the twobody "kernel diagrams" in these files are all using base
unit of $\MeV$.

\textbf{Output units:} The code is constructed such that if the kernel is
given in units of $MeV^{-n}$, then the variable \texttt{Result()} and its
output-to-file is in $\MeV^{(3-n)}$.

Multiplying by powers of $\texttt{HC}=197.\dots\;\MeV\fm$ translates into output of final MEs \texttt{Result()} into powers of $\fm$.

EXAMPLE: Compton has twobody kernel in $\MeV^{-4}$ ($n=4$), so the resultant
ME is in $\MeV^{-1}$. To convert to $\fm$, multiply by \texttt{HC}. In Compton, that multiplication by HC is not done in the fortran code, but
later in the mathematica processing files.

EXAMPLE: Pion Photoproduction kernel has units $\MeV^{-2}$ if the output
should be the twobody functions $F_{TL}$. Therefore,  $n=-2$ and the
\texttt{Results()} output is in $\MeV^1$. But the $F_{TL}$ output should be in $\fm^{-1}$, so divide here in kernel by \texttt{HC} to get $\fm^{-1}$ units in \texttt{Results()}.

Routines which compute vectors, like \texttt{calcmomenta.f}, simply use the same base unit in and out, i.e. are "unit neutral".
 
Therefore, the overall units of the output are $\MeV^{3-n}$ for a kernel with base units $\MeV^{-n}$!

The conversion $\frac{(\texttt{HC})^3}{(2\pi)^3}$ above is programmed
in \texttt{finalstatesums.twobodyvia2Ndensity.f}'s "fact=...") and converts
between the $\fm$ units of the "mantle" and the $\MeV$ units of the "kernel".

It ALSO includes ONE of the Fourier volumes $\frac{1}{(2\pi)^3}$ as a factor of
the twobody integration. There is no second Fourier volume (killed by phase
space). This guarantees that onebody and twobody have the same size and can
simply be summed to get the total amplitude:
\begin{equation}
  \label{eq:addamps}
  \mbox{amplitude } = \mbox{ onebody } + \mbox{ twobody }\;\;,
\end{equation}
without any relative factors (provided both provide output in same base
units). However, \emph{to include this factor or not depends on your choice of
  twobody kernel}.

That means in pion photoproduction, one may instead define it as part of the
prefactor \texttt{K2n} of a diagram. This implies that if you want the twobody
\texttt{Result()} output to be $F_{TL}$, you must un-compensate it here by
\texttt{*(2*Pi)**3}. On the other hand, if you want the twobody
\texttt{Result()} output to be $E_+$ etc., so that you can simply add as in
eq.~\eqref{eq:addamps} above, then the prefactor \texttt{K2n} shouldNOT
contain the $\frac{1}{(2\pi)^3}$, i.e.~multiply NOT with
\texttt{*(2*Pi)**3/HC}, but with
\begin{equation}
  \texttt{K2n = sqrt(4*Pi*alpaEM)*gA*mpi**2/(16*Pi*fpi**3)*10**3}
\end{equation}
to get \texttt{Result()} in the canonical units of $10^{-3}\;m_{\pi+}^{-1}$.
 
Set your kernel up in \texttt{2Bkernel.*.f} so that your \texttt{Result()} has the desired units and factors of $(2\pi)^3$. Do NOT make unit changes outside this file!



%%%%%%%%%%%%%%%%%%%%%%%%%%%%
\section{Integration Coding}
\label{sec:integration}

\subsection{Overall Structure}
\label{sec:integration-structure}

The integral is symbolically (including the permutation factor already
in the code itself, and denoting on the left the units of each line):
\begin{equation}
  \label{eq:twobody}
  \begin{split}
    [\MeV]^{3-n}\hqq&\;\genfrac{(}{)}{0pt}{}{A}{2}\;
    \bra\wfbra \sep M ' \, | O_{12} | \wf \sep M \, \ket \equiv\\
  % \\[0.5ex] &
    [\fm]^{-6}\hqq&\hqqq\genfrac{(}{)}{0pt}{}{A}{2}
    \sum_{\genfrac{}{}{0pt}{}{\fs\texttt{mt12,j12,s12,l12,m12}}{\fs
  \text{in \texttt{main.twobody.f}}}}
  \sum_{\genfrac{}{}{0pt}{}{\fs
      \texttt{\pbox{\linewidth}{\hf mt12p=mt12\hf\\[-2ex]j12p,s12p,l12p\\[-2ex]\hf m12p,Mzp,Mz\hf}}}
    {\fs\text{in \texttt{finalstatesums.*.f}}}}
  \underbrace{\int\frac{\deint{}{p_{12}'} p_{12}^{\prime \, 2 }}{(2\pi)^3}\;\int
    \deint{}{p_{12}} p_{12}^{2}}_{\text{in \texttt{fact=\dots} of \texttt{finalstatesums.*.f}}}\;\\
  [\MeV\fm]^3\hqq&\hqqq\times
%  O_{12}^{\alpha_{12}' \alpha_{12}} (p_{12}', p_{12}) \;
  \texttt{HC}^3\hspace*{25ex}\mbox{in \texttt{fact=\dots} of \texttt{finalstatesums.*.f}}\\
  [\fm]^3\hqq&\hqqq\times\rho_{\alpha_{12}'\,\alpha_{12}}^{%M_{T},
    M'M} (p_{12}',p_{12};\qv)\hspace*{10.8ex}\mbox{in \texttt{fact=\dots} of \texttt{finalstatesums.*.f}}\\
  [\text{no units}]\hqq&\hqqq\times\hqmmm\hqmmm\hqmm\sum_{\genfrac{}{}{0pt}{}{\fs
    \texttt{msp}\equiv m_{12}^{s\prime},\texttt{ms}\equiv m_{12}^s}{\fs\text{in \texttt{calculate2BI2.f}}}}
  \hspace*{-5ex}\CG{l_{12}'}{s_{12}'}{\jrel'}
  {m_{12}' - m^{s \prime }_{12}}{m^{s
      \prime}_{12}}{m_{12}'} \CG{l_{12}}{s_{12}}{\jrel}
  {m_{12} - m^{s}_{12}}{m^{s}_{12}}{m_{12}}   \hf\\[0.5ex]
  [\text{no units}]\hqq&\hqqq\hqqq\times \hqmmm\int\limits_{\fs\text{in \texttt{calculate2BI2.f}}} \hqmmm\deint{}{\hat p_{12}'} \deint{}{\hat p_{12}}
  Y^{\dagger}_{l_{12}' (m_{12}' - m^{s \prime }_{12})} ( \hat p_{12}') \;
  Y_{l_{12} (m_{12} - m^{s }_{12})} ( \hat p_{12})\; \\
  [\MeV]^{-n}\hqq&\hqqq\hqqq\hqqq\hqqq\times \underbrace{O_{12}^{\alpha_{12}'m_{12}^{s\prime} \alpha_{12}m_{12}^{s}} (\pv_{12}^{\,\prime},
    \pv_{12}; \qv )}_{\text{in \texttt{2Bkernel.*.f}}} \;\;.
\end{split}
\end{equation}
using the Clebsch-Gordan coefficients $\CG{j_{1}}{j_{2}}{j}{m_1}{m_{2}}{m}$ in
the convention of refs.~\cite{Edmonds, PDG}. The assignation of quantum
numbers between code and the quantities should be self-explaining except maybe
$\texttt{ms}=m_{12}^s$, $\texttt{msp}=m_{12}^{s\prime}$. The isospin
$\texttt{t12}=t_{12}^\prime$ is determined by the Pauli principle:
$(-)=(-)^{s_{12}+l_{12}+t_{12}}=(-)^{s_{12}^\prime+l_{12}^\prime+t_{12}^\prime}$, implemented in \texttt{main.twobody.f} for
the $\texttt{t12}\in\{0;1\}$ case as $\texttt{t12=(1-(-1)**(l12+s12+1))/2}$.

Only the last line is coded in the kernel's \texttt{2Bkernel.*.f} and
\texttt{spintricks*.f}, namely the twobody operator $\calO_{12}$ projected
onto orbital angular momenta via Clebsches.

All other lines are coded in the mantle.

%%%%%%%%%%%%%%%%%%%%%%%%%%%%%%%%%%%%%%%%%%%%
\subsection{Radial Integrations}
\label{sec:integration-radial}

The radial integration grids are set up in \texttt{main.twobody.f}
via

\texttt{call TRNS(NP12A,NP12B,NP12,P12A,P12B,P12C,P12MAG,AP12MAG)}

They are passed down to \texttt{finalstatesums.*.f}, where they are performed:

\begin{tabular}{ll}
  $\texttt{ip12}=1,\dots,\texttt{NP12}$& index of the \texttt{NP12} points of radial integration $p_{12}$\\
  \texttt{P12MAG(ip12)}& momentum  $p_{12}$ of index \texttt{ip12}\\
  \texttt{AP12MAG(ip12)}& Gau\3ian weight of momentum  $p_{12}$ of index \texttt{ip12}
\end{tabular}

\absatz
The integration over $p_{12}$ is done as sum over \texttt{ip12} in
\texttt{main.twobody.f}, but the integrand (including weight
$p_{12}^2\texttt{AP12MAG}$) is set up in \texttt{finalstatesums.*.f}.

\absatz
The integration over $p_{12}^\prime$ is
done inside \texttt{finalstatesums.*.f} as sum over \texttt{ip12p}:

\begin{tabular}{ll}
  $\texttt{ip12p}=1,\dots,\texttt{NP12}$& index of the \texttt{NP12} points of radial integration $p_{12}^\prime$\\
  \texttt{P12MAG(ip12p)}& momentum  $p_{12}^\prime$ of index \texttt{ip12p}\\
  \texttt{AP12MAG(ip12p)}& Gau\3ian weight of momentum  $p_{12}^\prime$ of index \texttt{ip12p}
\end{tabular}

\absatz
The combined integrand plus permutation factor plus measure plus density plus
$\hbar c$ factor is set up in \texttt{finalstatesums.*.f} as
\begin{equation}
\begin{split}
    \texttt{fact=}&
    \underbrace{\texttt{Anucl*(Anucl-1)/2}}_{\dis\genfrac{(}{)}{0pt}{}{A}{2}}
    \underbrace{\texttt{*p12**2*wp12}}_{\dis\deint{}{p_{12}} p_{12}^{2}}\\&    \underbrace{\texttt{*P12MAG(ip12p)**2*AP12MAG(ip12p)/(2*Pi)**3}}_{\dis\frac{\deint{}{p_{12}'} p_{12}^{\prime \, 2 }}{(2\pi)^3}}\\
  &\underbrace{\texttt{*rho(ip12,ip12p,rindx)}}_
  {\dis\rho_{\alpha_{12}'\,\alpha_{12}}^{M'M} (p_{12}',p_{12};\qv)}
  \underbrace{\texttt{*HC**3.d0}}_{\dis(\hbar c)^3}
  \end{split}
\end{equation}

%%%%%%%%%%%%%%%%%%%%%%%%%%%%%%%%%%%%%%%%%%%%
\subsection{Angular Integrations}
\label{sec:integration-radial}

The angular  integration grids are set up in \texttt{main.twobody.f}
via

\texttt{call Setquad12(th12,Nth12,phi12,Nphi12,\dots)}

and passed via \texttt{finalstatesums.*.f} to \texttt{calculate2BI2.f}, where
they are performed, calling the kernel-specific file \texttt{2Bkernel.*.f}.

The integral is performed by adding iteratively to the variable
\texttt{Int(extQnum,ml12p,ml12)} the following:
\begin{equation}
  \begin{split}
    \underbrace{\texttt{Yl12pstar(ml12p)}}_
    {\dis Y^{\dagger}_{l_{12}' (m_{12}' - m^{s \prime }_{12})} ( \hat p_{12}')}
    \underbrace{\texttt{*Yl12(ml12)}}_{\dis Y_{l_{12} (m_{12} - m^{s }_{12})} ( \hat p_{12})}&
 \underbrace{\texttt{*angweight12(ith,iphi)}}_
 {\dis\deint{}{\hat p_{12}}}
 \underbrace{\texttt{*angweight12(jth,jphi)}}
 _{\dis\deint{}{\hat p_{12}^\prime}}\\[2ex]&
 \underbrace{\texttt{*Kernel2B(extQnum,s12p,msp,s12,ms)}}_
 {\dis O_{12}^{\alpha_{12}'m_{12}^{s\prime} \alpha_{12}m_{12}^{s}} (\pv_{12}^{\,\prime}, \pv_{12}; \qv )}
     \end{split}
   \end{equation}
   [Strictly speaking,
   \texttt{Yl12pstar=Real(Yl12p(ml12p))-ci*Imag(Yl12p(ml12p))} is defined just
   before the summation \texttt{do extQnum=1,extQnumlimit} is done.]

   \noindent
   [\texttt{ml12=m12-ms}, \texttt{ml12p=m12p-msp}. The other quantum numbers
   should be self-explanatory.]
   
The summation is again over indices:

\begin{tabular}{ll}
  $\texttt{ith}=1,\dots,\texttt{Nth12}$& index of the \texttt{Nth12} points of
                                        angular integration $\theta_{12}$\\
  $\texttt{iphi}$& index of the \texttt{Nphi12} points of
                                          angular integration $\phi_{12}$\\
  & where $\texttt{iphi=ith}$ when the Lebedev-Laikov method is used\\
  & \phantom{where} $\texttt{iphi=1,\dots,\texttt{Nphi12}}$ when Gau\3ian integrations\\&\hf
    in $\theta$ and $\phi$ separately are used\\
  \texttt{th12(ith)}& angle  $\theta_{12}$ of index \texttt{ith}\\
  \texttt{phi12(iphi)}& angle  $\phi_{12}$ of index \texttt{iphi}\\
  \texttt{angweight12(ith,iphi)}& weight of angle  $(\theta_{12},\phi_{12})$ of index (\texttt{ith},\texttt{iphi}).
\end{tabular}

\absatz
For the primed variables:

\begin{tabular}{ll}
  $\texttt{jth}=1,\dots,\texttt{Nth12}$& index of the \texttt{Nth12} points of
                                        angular integration $\theta_{12}^\prime$\\
  $\texttt{jphi}$& index of the \texttt{Nphi12} points of
                                          angular integration $\phi_{12}^\prime$\\
  & where $\texttt{jphi=jth}$ when the Lebedev-Laikov method is used\\
  & \phantom{where} $\texttt{jphi=1,\dots,\texttt{Nphi12}}$ when Gau\3ian integrations\\&\hf
    in $\theta$ and $\phi$ separately are used\\
  \texttt{th12(jth)}& angle  $\theta_{12}^\prime$ of index \texttt{jth}\\
  \texttt{phi12(jphi)}& angle  $\phi_{12}^\prime$ of index \texttt{jphi}\\
  \texttt{angweight12(jth,jphi)}& weight of angle  $(\theta_{12}^\prime,\phi_{12}^\prime)$ of index (\texttt{jth},\texttt{jphi}).
\end{tabular}


%%%%%%%%%%%%%%%%%%%%%%%%%%%%%%%%%%%%%%%%%%%%%%%%%%%%%%%%%%%%%%%%%%%%%%%%%%%%%%%
\section{}
\label{sec:}

%%%%%%%%%%%%%%%%%%%%%%%%%%%%
\subsection{}
\label{sec:}

%%%%%%%%%%%%%%%%%%%%%%%%%%%%%%%%%%%%%%%%%%%%%%%%%%%%%%%%%%%%%%%%%%%%%%%%%%%%%%%
\section{Concluding Questions}
\label{sec:conculsions}


% %%%%%%%%%%%%%%%%%%%%%%%%%%%%
% \subsection{}
% \label{sec:}

%%%%%%%%%%%%%%%%%%%%%%%%%%%%%%%%%%%%%%%%%%%%%%%%%%%%%%%%%%%%%%%%%%%%%%%%%%%%%%%
%%%%%%%%%%%%%%%%%%%%%%%%%%%%%%%%%%%%%%%%%%%%%%%%%%%%%%%%%%%%%%%%%%%%%%%%%%%%%%%
%%%%%%%%%%%%%%%%%%%%%%%%%%%%%%%%%%%%%%%%%%%%%%%%%%%%%%%%%%%%%%%%%%%%%%%%%%%%%%%

\section*{Acknowledgements}

As usual, we acknowledge no responsibility whatsoever.

\newpage

%%%%%%%%%%%%%%%%%%%%%%%%%%%%%%%%%%%%%%%%%%%%%%%%%%%%%%%%%%%%%%%%%%%%%%%%%%%%%%%
\appendix
% %%%%%%%%%%%%%%%%%%%%%%%%%%%%%%%%%%
% \section{}
% \label{app:appendix}

% %%%%%%%%%%%%%%%
% \subsection{}
% \label{app:}

% \newpage
%%%%%%%%%%%%%%%%%%%%%%%%%%%%%%%%%%%%%%%%%%%%%%%%%%%%%%%%%%%%%%%%%%%%%%%%%%%%%%%
%
% TAKE SPIRES LaTeX(US) STYLE
%
\begin{thebibliography}{99}

%\cite{Edmonds}
\bibitem{Edmonds} A.~R.~Edmonds, ``Angular Momentum in Quantum Mechanics'',
  Princeton University Press (1974).

%\cite{PDG}
\bibitem{PDG}
  R.~L.~Workman \textit{et al.} [Particle Data Group],
  % ``Review of Particle Physics,''
  PTEP \textbf{2022} (2022) 083C01
  \doi{10.1093/ptep/ptac097} and \url{http://pdg.lbl.gov}.
  % 3 citations counted in INSPIRE as of 12 Jul 2022

\end{thebibliography}

%%%%%%%%%%%%%%%%%%%%%%%%%%%%%%%%%%%%%%%%%%%%%%%%%%%%%%%%%%%%%%%%%%%%%%%%%%%%%%%
%\begin{equation}
%  \label{eq:}
%\end{equation}
%%%%%%%%%%%%%%%%%%%%%%%%%%%%%%%%%%%%%%
% use following instead of eqnarray (append "*" for no numbers) 
%%%%%%%%%%%%%%%%%%%%%%%%%%%%%%%%%%%%%%
%\begin{align}
%  \label{eq:}
%\end{align}
%%%%%%%%%%%%%%%%%%%%%%%%%%%%%%%%%%%%%%
% use following for one equation number over several lines:
%%%%%%%%%%%%%%%%%%%%%%%%%%%%%%%%%%%%%%
%\begin{equation}
%  \label{eq:}
%  \begin{split}
%  \end{split}
%\end{equation}
%%%%%%%%%%%%%%%%%%%%%%%%%%%%%%%%%%%%%%
%\begin{figure}[!htbp]
%\begin{center}
%  \ifx\feynVersion\AnswerYes
%      %%% if Feynman graphs explicit
%      \feyngraph{100}{100}{}{}{}{}{
%         ******}
%  \else
%      %%% if Feynman graphs as .eps file
%      \includegraphics[width=0.82\linewidth]{}
%  \fi
%\caption{(Colour online) }
%\label{fig:}
%\end{center}
%\end{figure}
%%%%%%%%%%%%%%%%%%%%%%%%%%%%%%%%%%%%%%
%\begin{figure}[!htbp]
%\begin{center}
%      \includegraphics[width=0.82\linewidth]{}
%\caption{(Colour online) }
%\label{fig:}
%\end{center}
%\end{figure}
%%%%%%%%%%%%%%%%%%%%%%%%%%%%%%%%%%%%%%
% \begin{figure}[!htbp]
% \begin{center}
%   \includegraphics[width=0.48\linewidth]{}\hfill
%   \includegraphics[width=0.48\linewidth]{}
% \caption{(Colour online) }
% \label{fig:}
% \end{center}
% \end{figure}
%%%%%%%%%%%%%%%%%%%%%%%%%%%%%%%%%%%%%%
% \begin{figure}[!htbp]
% \begin{center}
%   \includegraphics[width=0.48\linewidth]{}\hfill
%   \includegraphics[width=0.48\linewidth]{}
%   \\[2ex]
%   \includegraphics[width=0.48\linewidth]{}\hfill
%   \includegraphics[width=0.48\linewidth]{}
% \caption{(Colour online) }
% \label{fig:}
% \end{center}
% \end{figure}
%%%%%%%%%%%%%%%%%%%%%%%%%%%%%%%%%%%%%%
% \begin{table}[!htbp]
%   \centering
%   \begin{tabular}{|c||l|l||c|}
%     \hline
%    \rule[-1.5ex]{0ex}{4ex}
%    partial wave $l$&$s_l(\lambda=1)$&$s_l(\lambda=-\half)$
%    &$s_{l,\text{simplistic}}=l+1$\\
%    \hline
%    \hline
%    \rule[-1.5ex]{0ex}{4ex}
%    0&$1.00624\dots\;\ii$&2.16622\dots&1\\
%    \hline
%   \end{tabular}
%   \caption{} 
%   \label{tab:}
% \end{table}
%%%%%%%%%%%%%%%%%%%%%%%%%%%%%%%%%%%%%%%%%%%%%%%%%%%%%%%%%%%%%%%%%%%%%%%%%%%%%%%
\ifx\feynVersion\AnswerYes
   \end{fmffile}
\fi
\end{document}
%Nice you made it till here.
